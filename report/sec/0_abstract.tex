\begin{abstract}
실제 배포 환경에서는 학습(source)-배포(target) 간 도메인 편차로 인해 모델 성능이 급격히 저하된다. 본 연구는 이러한 문제를 해결하기 위해 통계적 정렬 기반 MMD, 주파수 증강 기반 FACT, Cosine Similarity 정렬 손실, 그리고 위상 지터와 그래프 일관성을 결합한 신규 기법 FPJ-GC를 ResNet-50 기반 분류기에 적용해 비교하였다. PACS 데이터셋으로 학습하고 동일 카테고리의 실사 이미지를 수집해 구축한 CuratedPACS로 성능을 평가한 결과, 기존 베이스라인(69.40 / 66.53) 대비 FACT(+11.5 \~ 10.7), Cosine Similarity(+12.0 \~ 12.3), FPJ-GC(+12.4 \~ 15.6)가 일관된 정확도 향상을 보였다. 특히 FPJ-GC는 추가 파라미터 없이 위상 기반 변형과 배치 전역 그래프 정합성을 동시에 활용해 가장 높은 평균 정확도(81.83 / 82.09)와 낮은 성능 분산을 기록했다.

\end{abstract}